% Created 2022-10-20 gio 22:18
% Intended LaTeX compiler: lualatex
\documentclass[letterpaper, 11pt]{article}

\usepackage{lmodern} % Ensures we have the right font
\usepackage[T1]{fontenc} % Basic font & characters selection
\usepackage[utf8]{inputenc}
\usepackage{fontspec}
% Define font family to use (Other options are: Iosevka, Source Code Pro, Ubuntu, Titillium -> see setup file)
\setmainfont{RobotoCondensed}[Extension=.ttf, UprightFont=*-Light, BoldFont=*-Regular, ItalicFont=*-LightItalic, BoldItalicFont=*-BoldItalic, Path=/home/valentino/Dropbox/fonts/]

% Tables, wrapping, and other options
\usepackage{longtable} % This package defines the longtable environment, a multi-page version of tabular
\usepackage{wrapfig} % This makes the wrapfigure environment available
\usepackage{rotating} % Pretty obvious
\usepackage[normalem]{ulem} % underlining and strike-through
\usepackage{capt-of} % Captions outside of floats
\usepackage{graphicx} % Include images
\usepackage{amsmath, amsthm, amssymb} % Subscript & superscript and math environments (amsmath),  Various symbols used for interpreting the entities (amssymb)
\usepackage[table, xcdraw]{xcolor}
\usepackage{listings} % Code highlighting
\usepackage{mdframed} % \usepackage[framemethod=TikZ]{mdframed} (Alternativa in caso si voglia usare Tikz come metodo)

% COLOR DEFINITION
\definecolor{classTIKZcolor}{RGB}{222,222,222}
% DEFINIZIONE COLORI TIKZ
\definecolor{darkblue}{RGB}{0,60,104}
\definecolor{darkdark}{RGB}{22,22,22}
% DEFINIZIONE COLORI DA USARE PER IL CODICE
\definecolor{airforceblue}{rgb}{0.36, 0.54, 0.66}

% Definizione colori per i linguaggi da usare
\definecolor{orangered}{RGB}{239,134,64}
\definecolor{includeStatementCPP}{RGB}{148,123,155}
\definecolor{libraryStatementCPP}{RGB}{126,190,184}
\definecolor{colorMainCPP}{RGB}{190,116,67}
\definecolor{colorTypesCPP}{RGB}{188,90,69}
\definecolor{colorReservedKeywordsCPP}{RGB}{130,183,75}
\definecolor{colorLoopsCPP}{RGB}{185,176,176}
\definecolor{colorOtherKeywordsCPP}{RGB}{254,178,54}
\definecolor{headerJava}{HTML}{006b3C}
\definecolor{packageNameJavaDefinition}{HTML}{007BA7}
\definecolor{classKeywordJava}{HTML}{CD5C5C}
\definecolor{classNameJava}{HTML}{D2691E}
\definecolor{methodKeyword}{HTML}{FE6F5E}
\definecolor{constantsKeyword}{HTML}{FFA812}
\definecolor{attributesKeyword}{HTML}{9955BB}
\definecolor{testKeyword}{HTML}{1E90FF}
\definecolor{assertKeyword}{HTML}{6082B6}
\definecolor{expectedKeyword}{HTML}{29AB87}
\definecolor{nullKeyword}{HTML}{E66771}
\definecolor{variableKeyword}{HTML}{778899}
% Definizione colori definitiva
\definecolor{orangelight}{RGB}{238,162,82}
\definecolor{greenlight}{RGB}{147,196,125}
\definecolor{purplelight}{RGB}{176,159,222}
\definecolor{bluelight}{RGB}{122,171,216}
\definecolor{redstrong}{RGB}{255,40,50}
\definecolor{greendark}{RGB}{144,161,106}
\definecolor{footerColor}{RGB}{0,163,243}
\definecolor{footerColorSurrounding}{RGB}{22,154,255}
\definecolor{ygroblue}{HTML}{179AFF}
% TIKZ
\usepackage{tikz}
% DEFINIZIONE CAMPI E FORME PER TIKZ
\usetikzlibrary{calc,shadows,shapes,arrows}

% DEFINIZIONE FORME GEOMETRICHE PER DIAGRAMMI
\tikzstyle{CIRCLE} = [circle, minimum width=0.8cm, minimum height=0.8cm,text centered, draw=black, fill=blue!30]
\tikzstyle{CIRCLESMALL} = [circle, minimum width=0.2cm, minimum height=0.2cm,text centered, draw=black, fill=black]
\tikzstyle{startstop} = [rectangle, rounded corners, minimum width=3cm, minimum height=1cm,text centered, draw=black, fill=red!30]
\tikzstyle{io} = [trapezium, trapezium left angle=70, trapezium right angle=110, minimum width=3cm, minimum height=1cm, text centered, draw=black, fill=blue!30]
\tikzstyle{process} = [rectangle, minimum width=3cm, minimum height=1cm, text centered, text width=3cm, draw=black, fill=orange!30]
\tikzstyle{decision} = [diamond, minimum width=3cm, minimum height=1cm, text centered, draw=black, fill=green!30]
\tikzstyle{class}=[rectangle, draw=black, text centered, anchor=north, text=black, text width=3cm, shading=axis, bottom color=classTIKZcolor,top color=white,shading angle=45]
\tikzstyle{arrow} = [thick,->,>=stealth]


% OPTIONS FOR MDFRAMES (ENVIRONMENT)
\makeatletter
\mdfdefinestyle{@mdf@stubenv}{
leftmargin=2pt,
rightmargin=2pt,
innermargin=0pt,
outermargin=0pt,
skipabove=2pt,
skipbelow=2pt,
linewidth=1pt,
linecolor=stub@tmp!80!black,
frametitlebackgroundcolor=stub@tmp!80!black,
backgroundcolor=stub@tmp,
innertopmargin=1pt,
innerbottommargin=1pt,
innerleftmargin=1pt,
innerrightmargin=1pt,
nobreak=true}
\def\make@stubenv#1#2#3#4{
\global\newcounter{stub@#1}[#4]
\newenvironment{#1}[1][]{
\colorlet{stub@tmp}{#2!25}
\begin{mdframed}[style=@mdf@stubenv, frametitle={\scriptsize \% #3 ##1}]
}{
\global\stepcounter{stub@#1}\end{mdframed}
}
}
\make@stubenv{definition}{ygroblue}{}{subsection}
\makeatother

% SET DEFAULT OPTION FOR CODE BLOCKS
\lstset {
frame=trBL, %frame=single
framesep=\fboxsep,
framerule=\fboxrule,
frameround=fttt,
rulecolor=\color{black},
xleftmargin=\dimexpr\fboxsep+\fboxrule,
xrightmargin=\dimexpr\fboxsep+\fboxrule,
breaklines=true,
basicstyle=\small\tt,
keywordstyle=\color{blue}\sf,
columns=flexible,
}

% Settaggio stile per i linguaggi da usare
\lstdefinestyle{CPP}{
language=C++,
backgroundcolor=\color{white},
escapeinside={`'},
numbers=left,
numbersep=15pt,
numberstyle=\tiny,
commentstyle=\color{gray},
% #include statement
keywords=[1]{\#include},
keywords=[2]{ main },
% types
keywords=[3]{int, char, short, long, float, double},
% List of reserved keywords
keywords=[4]{auto, struct , unsigned, signed, enum, register, typedef, extern, return, union, continue, goto, volatile, default, define, static},
% Loops
keywords=[5]{do, while, case, else, switch, break, for, if},
% List of other keywords
keywords=[6]{void, boolean, const, sizeof, sleep},
% Colors of the keywords:
keywordstyle=[1]\color{includeStatementCPP},
keywordstyle=[2]\color{colorMainCPP},
keywordstyle=[3]\color{colorTypesCPP},
keywordstyle=[4]\color{colorReservedKeywordsCPP},
keywordstyle=[6]\color{colorOtherKeywordsCPP}
}
\lstdefinestyle{BASH}{
language=bash,
commentstyle=\color{gray},
backgroundcolor=\color{white},
numbers=left,
numbersep=15pt,
numberstyle=\tiny,
stringstyle=\color{greendark},
commentstyle=\color{gray},
keywords=[1]{exit, print_error, fail},
keywords=[2]{printf, cut, print_ok, basename, while, usage, run},
keywords=[3]{ls, find, touch, egrep, print_info, read, done},
keywords=[4]{wc},
keywordstyle=[1]\color{redstrong},
keywordstyle=[2]\color{greenlight},
keywordstyle=[3]\color{bluelight},
keywordstyle=[4]\color{purplelight}
}
\lstdefinestyle{JAVA}{
language=Java,
backgroundcolor=\color{white},
numbers=left,
numbersep=15pt,
numberstyle=\tiny,
commentstyle=\color{gray},
stringstyle=\color{gray},
keywords=[1]{package, import, static, public, return, true},
keywords=[2]{android, Manifest, content, Context, util, Log, androidx, core, app, ActivityCompat, org, test, platform, app, InstrumentationRegistry, ext, junit, runner, runners, AndroidJUnit4, Test, RunWith, Assert, com, natour, utils, constants, Constants, persistence, LocalUser, LocalUserDbManager, java, regex, Pattern, Before},
keywords=[3]{void, boolean, int, String, while, synchronized, volatile, long, double},
keywords=[4]{class, @RunWith, super},
keywords=[5]{if, Employee, try, catch},
keywords=[6]{localUser, dbManager, checkFineLocation, checkCoarseLocation, appContext, controlloRecuperoPassword},
keywords=[7]{private},
keywords=[8]{@Test, @Before},
keywords=[9]{assertTrue, assertFalse},
keywords=[10]{expected, IllegalArgumentException},
keywords=[11]{null, false},
keywords=[12]{username, email, password, confermaPassword, pattern},
keywordstyle=[1]\color{headerJava},
keywordstyle=[2]\color{packageNameJavaDefinition},
keywordstyle=[3]\color{methodKeyword},
keywordstyle=[4]\color{classKeywordJava},
keywordstyle=[5]\color{classNameJava},
keywordstyle=[6]\color{attributesKeyword},
keywordstyle=[7]\color{constantsKeyword},
keywordstyle=[8]\color{testKeyword},
keywordstyle=[9]\color{assertKeyword},
keywordstyle=[10]\color{expectedKeyword},
keywordstyle=[11]\color{nullKeyword},
keywordstyle=[12]\color{variableKeyword},
morekeywords={*,...}
}
\lstdefinestyle{XML}{
language=XML,
backgroundcolor=\color{white},
numbers=left,
numbersep=15pt,
numberstyle=\tiny,
commentstyle=\color{gray},
keywords=[1]{article, author, title, description, text, formula, math},
keywords=[2]{dbs:,dbs:book,<dbs:description>},
keywordstyle=[1]\color{colorMainCPP},
keywordstyle=[2]\color{includeStatementCPP},
}

% Colorizing links in a nicer way.
\usepackage{hyperref} % Links
\hypersetup{colorlinks, linkcolor=black, urlcolor=blue}
%Bibliography (Not added yet)
\usepackage[backend=biber,sortcites,style=verbose-trad2]{biblatex}
\bibliography{./Bibliography.bib}
\usepackage{titling}
\setlength{\droptitle}{-9em}
\setlength{\parindent}{0pt}
\setlength{\parskip}{1em}
\usepackage[stretch=10]{microtype}
\usepackage{hyphenat}
\usepackage{ragged2e}
\usepackage{subfig} % Subfigures (not needed in Org I think)
\usepackage[top=1in, bottom=1.25in, left=0.55in, right=0.55in]{geometry} % Page geometry
\renewcommand{\baselinestretch}{1.15}
% Page numbering - Footer
\usepackage{fancyhdr} % Custom headers and footers
\pagestyle{fancy} % Makes all pages in the document conform to the custom headers and footers
\fancyhead{} % No page header
\renewcommand{\headrulewidth}{0pt}
\fancyfoot[L]{} % Empty left footer
\fancyfoot[C]{} % Empty center footer
\newcommand\FrameBoxR[1]{
\fcolorbox{footerColorSurrounding}{footerColor}{\makebox[3cm][r]{\textcolor{white}{\bfseries#1}}}
}
\fancyfoot[R]{\FrameBoxR{\thepage}}
\usepackage[explicit]{titlesec}

% Title customization
\pretitle{\begin{center}\fontsize{20pt}{20pt}\selectfont}
\posttitle{\par\end{center}}
\preauthor{\begin{center}\vspace{-6bp}\fontsize{14pt}{14pt}\selectfont}
\postauthor{\par\end{center}\vspace{-25bp}}
\predate{\begin{center}\fontsize{12pt}{12pt}\selectfont}
\postdate{\par\end{center}\vspace{0em}}

% Section/subsection headings:
%Section
\titlespacing\section{0pt}{2pt}{2pt} % left margin, space before section header, space after section header
%Subsection
\titlespacing\subsection{0pt}{5pt}{-2pt} % left margin, space before subsection header, space after subsection header
%Subsubsection
\titlespacing\subsubsection{0pt}{5pt}{-2pt} % left margin, space before subsection header, space after subsection header

% List spacing & options
\usepackage{enumitem}
\usepackage{fdsymbol}
\setlist{itemsep=-2pt} % or \setlist{noitemsep} to leave space around whole list
\setlist[enumerate, 1]{label=\arabic*.}
\setlist[enumerate, 2]{label=\Roman*.}
\setlist[enumerate, 3]{label=\alph*.}
\setlist[itemize, 1]{label=$\smallblacktriangleright$}
\setlist[itemize, 2]{label=$\smalldiamond$}
\setlist[itemize, 3]{label=$\smallcircle$}
\author{Valentino Bocchetti}
\date{}
\title{}
\begin{document}

\thispagestyle{empty}
\newgeometry{margin=0pt}

\begin{tikzpicture}[remember picture, overlay]
  \begin{scope}

    % STRUTTURA ESTERNA (ANGOLI E COLORAZIONE)
    \node[
      isosceles triangle,
      isosceles triangle apex angle=90,
      draw,
      rotate=315,
      fill=darkblue,
      minimum size =55cm] (triangoloPrimoLivelloAngoloInferiore)
    at ($(current page.south east)$)
    {};

    \node[
      isosceles triangle,
      isosceles triangle apex angle=90,
      draw,
      rotate=315,
      fill=footerColorSurrounding,
      minimum size =50cm] (triangoloSecondoLivelloAngoloInferiore)
    at ($(current page.south east)$)
    {};

    \node[
      isosceles triangle,
      isosceles triangle apex angle=90,
      draw,
      rotate=315,
      fill=darkdark,
      minimum size =35cm] (triangoloTerzoLivelloAngoloInferiore)
    at ($(current page.south east)$)
    {};

    \node[
      isosceles triangle,
      isosceles triangle apex angle=90,
      draw,
      rotate=315,
      fill=white,
      minimum size =32cm] (triangoloQuartoLivelloAngoloInferiore)
    at ($(current page.south east)$)
    {};


    \node[
      isosceles triangle,
      isosceles triangle apex angle=90,
      draw,
      rotate=135,
      fill=white,
      minimum size =31cm] (triangoloSuperiore)
    at ($(current page.north west)$)
    {};


    % STRUTTURA DEL CERCHIO E IL SUO CONTENUTO
    \node [circle, minimum size=15cm, fill=white, draw=darkblue, line width = 7pt, xshift=11cm, yshift=1cm](centro)
    at ($(current page.west)$)
    {};

    \coordinate (logo) at (10.7,-7.5);
    \node[rectangle, inner sep = 0pt, outer sep = 0pt, minimum width = 3.5cm, minimum height = 3.5cm]
    at (logo){\includegraphics[width=3.5cm, height=3.5cm]{./Risorse/Start-Page/FedericoII.png}};

    \node[scale=1.5] at (5.85,-10){\LARGE{StealBot}}; %% TODO: Pensare ad una alternativa

    \node[rectangle,
      draw,
      minimum width=4cm,
      minimum height=2mm,
      xshift= 6.42cm,
      yshift= -11cm,
      minimum height=2mm,
      fill = darkblue] (r) at (0,0) {};
 
    \node[scale=1.5] at (11,-13){\itshape{\LARGE{Università degli studi di Napoli}}};
    \node[scale=1.5] at (11,-15){\itshape{\LARGE{Federico II}}};


    %% STRUTTURA ANGOLO SINISTRO (angolo north-west)
    \node[scale=3] at (1,-1){\includegraphics[height=10pt,width=10pt]{./Risorse/Start-Page/calendar.png}};
    \node[scale=2] at (6,-1){\itshape{\LARGE{A.A. 2022-2023}}};

    %% STRUTTURA ANGOLO DESTRO (angolo south-east) - Informazioni studenti
    %% Valentino Bocchetti
    \node[scale=3] at (12,-22){\includegraphics[height=10pt,width=10pt]{./Risorse/Start-Page/graduated.png}};
    \node[scale=1.5] at (16.5, -22.23){\textbf{Valentino Bocchetti - N86003405}};

    %% Valentina Annunziata
    \node[scale=3] at (12,-24){\includegraphics[height=10pt,width=10pt]{./Risorse/Start-Page/graduated_alt.png}};
    \node[scale=1.5] at (16.7, -24.23){\textbf{Valentina Annunziata - N86003280}};

    %% Francesco Ciccarelli
    \node[scale=3] at (12,-26){\includegraphics[height=10pt,width=10pt]{./Risorse/Start-Page/graduated.png}};
    \node[scale=1.5] at (16.58, -26.23){\textbf{Francesco Ciccarelli - N86003285}};

    %\draw [red,thick,domain=0:90, xshift=-4cm, yshift=3.5cm] plot ({cos(\x)}, {sin(\x)});
    %\draw [blue,thick,domain=180:270, xshift=-4cm, yshift=3.5cm] plot ({cos(\x)}, {sin(\x)});
  \end{scope}

\end{tikzpicture}

\newpage
\restoregeometry
\renewcommand*\contentsname{\hfill Indice \hfill}
\tableofcontents
\pagebreak
\section{Revisioni}
\label{sec:org41ea8be}
\begin{center}
\begin{tabular}{|p{0.1\textwidth}|p{0.1\textwidth}|p{0.2\textwidth}|p{0.5\textwidth}|}
\hline
Versione & Data & Autore & Descrizione\\
\hline
0.0.1 & 19-10-2022 & Valentino Bocchetti & Creazione della struttura del documento e strutturazione\\
 &  & Valentina Annunziata & dell'albero di lavoro\\
 &  & Francesco Ciccarelli & \\
\hline
\end{tabular}
\end{center}
\pagebreak
\section{Presentazione}
\label{sec:org38467b1}
\noindent\makebox[\textwidth]{\includegraphics[width=\paperwidth]{Risorse/Title.png}}
\subsection{Descrizione della traccia}
\label{sec:org8a1783d}
Si richiede la realizzazione di una \texttt{BotNET} \autocite{BOTNET} per il recupero di quante più informazioni possibili sulla dispositivo in cui una delle componenti della BotNET (a scelta dello studente) venga eseguito.
\subsubsection{Tecnologie e linguaggi richiesti}
\label{sec:orgdeb7f0a}
Si richiede un applicativo scritto in \texttt{Python} \autocite{DefinizionePython} che utilizzi come strumento di comunicazione le \texttt{socket} \autocite{DefinizioneSocket}
\subsection{Implementazione del sistema}
\label{sec:org82b1877}
Il progetto si concretizza in 2 componenti ben definite:
\begin{itemize}
\item Un \texttt{Bot Master} per la gestione dei dati ricevuti dal \texttt{bot slave} al quale inpartisce comandi sfruttando una connessione tramite socket asincrona;
\item Il \texttt{Bot slave}, che ha il compito di ricavare quante più informazioni possibili sullo stato della macchina sul quale viene eseguito \autocite{notaBotSlave1} .
\end{itemize}
\subsection{Guida al Bot Master}
\label{sec:orgd490e0a}
\subsubsection{Primo avvio}
\label{sec:org47c5c2a}
DESCRIZIONE DELLE OPERAZIONI EFFETTUATE IN FASE DI AVVIO
\subsubsection{Analisi della struttura del progetto}
\label{sec:org457c93f}
\subsubsection{Memorizzazione dei dati}
\label{sec:orgae7177c}
Il sistema permette inoltre utilizza un DBMS \autocite{postgres} per il salvataggio dei dati ricavati dal \emph{bot slave} durante la sua esecuzione.
\subsection{Guida al Bot Slave}
\label{sec:org53fd9b2}
\subsubsection{Analisi della struttura del progetto}
\label{sec:orged238f9}
\subsubsection{Primo avvio}
\label{sec:org74e5085}
DESCRIZIONE DELLE OPERAZIONI EFFETTUATE IN FASE DI AVVIO
\subsection{Report dei dati recuperati}
\label{sec:org029d861}
TODO: Aggiungere screenshot/tabella dei record ottenuti mediante il bot
\section{Dettagli implementativi}
\label{sec:org79de2f8}
\subsection{Bot Master}
\label{sec:org4042a74}
\subsection{Bot Slave}
\label{sec:org7151ce1}
\section{Codice sorgente sviluppato}
\label{sec:org24a2a7d}
Il codice sorgente prodotto durante lo sviluppo di \(StealBot^{\copyright}\) è disponibile sulla piattaforma \href{https://github.com/}{GitHub}, che ne ha permesso anche il versionamento.

Essendo molto vicini al concetto di open-source, si è scelto fin da subito un tipo di licenza che fosse in linea con questo principio. È per questo che tutti i programmi necessari alla piattaforma \emph{NaTour} sono sotto licenza \href{https://www.gnu.org/licenses/gpl-3.0.en.html}{GPL 3.0} (che è possibile consultare anche nella repository di NaTour).

Di seguito riportiamo un link per il \href{https://github.com/luftmensch-luftmensch/StealBot}{download} \autocite{informazioniRepository}
\section{Analisi dei tempi di sviluppo}
\label{sec:org1e1d539}
Di seguito viene riportata lo storico dei tempi di sviluppo sotto forma di timeline con una vista i passaggi chiave dello sviluppo di \(StealBit^{\copyright}\)
\section{Ringraziamenti}
\label{sec:org41171ee}
Ringraziamo il professore Alessio Botta per lo splendido corso, che ci ha permesso di comprendere a pieno tecnologie di tutti fanno largo uso.
\end{document}